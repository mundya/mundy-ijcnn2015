\documentclass[t]{beamer}
\usetheme[cabin, darktitle]{UniversityOfManchester}

%% Document properties
\title{An efficient SpiNNaker implementation of the NEF}
\author{Andrew Mundy, James Knight, Terry Stewart and Steve Furber}

\begin{document}
  \maketitle

  %% Introduction to the NEF
  %% - Encoding
  %% - Decoding
  %% - Forming weight matrices

  %% Constraints on SpiNNaker programs
  %% - Memory usage
  %%  - In its own right
  %%  - Time to load data
  %% - Processing time
  %%  - Dominated by synaptic weight lookup -- < 5000 synaptic events per timestep

  %% How does the NEF meet these constraints?
  %% - Consider communication channel using Spaun-like parameters of 16-D and 70 neurons per dimension
  %%  - Memory use is insane (down to 53 neurons per core)
  %%  - Synaptic events very quickly pass the max allowed
  %% Result: using the standard approach with the NEF will lead to suboptimal use of the architecture.

  %% An alternative: value-based transmission
  %% - Note that weight matrices can be factored into encoders and decoders
  %% - Indicate how this can be used in simulation using the payloads of MC packets

  %% Results
  %% - Memory usage
  %% - Processor usage

  %% Discussion (brief)

  %% Thanks!
\end{document}
